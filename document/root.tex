\documentclass[11pt,a4paper]{article}
\usepackage{isabelle,isabellesym}
\usepackage{amsfonts, amsmath, amssymb}

% this should be the last package used
\usepackage{pdfsetup}
\usepackage[shortcuts]{extdash}

% urls in roman style, theory text in math-similar italics
\urlstyle{rm}
\isabellestyle{it}

\begin{document}

\title{The Hermite--Lindemann--Weierstraß Transcendence Theorem}
\author{Manuel Eberl}
\maketitle

\begin{abstract}
This article provides a formalisation of the Her\-mite\--Lin\-de\-mann\--Wei\-er\-straß Theorem
(also known as simply Her\-mite\--Lin\-de\-mann or Lin\-de\-mann\--Wei\-er\-straß). This theorem is one of the
crowning achievements of 19th century number theory.

The theorem states that, for any algebraic numbers $\beta_1, \ldots, \beta_n\in\mathbb{C}$ and
distinct algebraic numbers $\alpha_i, \ldots, \alpha_n\in\mathbb{C}$, we have:
\[\beta_1 e^{\alpha_1} + \ldots + \beta_n e^{\alpha_n} = 0 \quad\quad\text{iff}\quad\quad
  \forall i.\ \beta_i = 0\]

This has a number of direct corollaries, e.g.:
\begin{itemize}
\item $e$ and $\pi$ are transcendental
\item $e^z$, $\sin z$, $\tan z$, etc.\ are transcendental for algebraic $z\in\mathbb{C}\setminus\{0\}$
\item $\ln z$ is transcendental for algebraic $z\in\mathbb{C}\setminus\{0, 1\}$
\end{itemize}

Like the previous formalisation in Coq by Bernard~\cite{bernard}, I formalised Baker's version
of the theorem and proof~\cite{baker}.
\end{abstract}

\newpage
\tableofcontents
\newpage
\parindent 0pt\parskip 0.5ex

\input{session}

\nocite{baker}
\nocite{redheffer_steinberg}
\nocite{bernard}

\bibliographystyle{abbrv}
\bibliography{root}

\end{document}

%%% Local Variables:
%%% mode: latex
%%% TeX-master: t
%%% End:
